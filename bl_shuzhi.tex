\documentclass[a4paper]{article}
\usepackage{ctex}

\title{百联全渠道2024年年底述职发言稿}
\author{hmy}
\date{2025-01-09}

\begin{document}
\maketitle
\newpage
各位领导大家好,我是大数据的胡梦源,非常荣幸在这里进行奇兵计划
述职。
我于2023年3月入职,岗位是助理分析师,接下来我将从岗位认知、
业绩成果、自我评价和未来规划四个方面展开。
\par

首先从岗位认知开始,我对标了大厂关于数据分析师岗位的描述
从中提炼出五点要求分别是

结合上述五点以及对近两年来工作的理解,我认为从事该岗位是需要 以业务知识为基础,以决策优化为目的,从数据出发,搭建指标体系,依靠各种数据分析方法,洞察业务背后的规律,为业务创造最大价值。

在实际工作的时候,为应对各种需求,我总结了一套自己的工作流。首先面对需求首先与业务沟通,理解业务同事在提出需求时所考虑的角度
从而提炼出具体问题,其次拆解需求获取数据进行数据清洗,根据具体的需求形势来得到结果,可能是简单的数据也有可能是BI看板,
最后反馈业务,获取业务的观点对业务需求产生进一步的理解,从而形成一个良性循环。

商业的本质是盈利,作为数据分析师大部分产出都无法盈利,


公司战略发展层面的业务能力;数据分析需要的最基础的数理统计能力;高效且出色完成工作所需要的python excel等软件或编程能力;和对接不同部门以及呈现结果的沟通与展示能力

从我个人的理解上,数据分析师属于工程技术岗,但与其他技术岗的区别是
软件编程等技术技能决定数据分析师工作的下限,数理统计和业务能力则是数据分析师职业发展的上限

就我个人目前所处的职业阶段来考虑,首先我应该保证基本的技术如sql、excel、python等能力过关,能够响应业务需求,然后结合已具备的数理统计知识,完成业务报表和小型数据分析项目,在此基础上,深刻理解业务,提升沟通与可视化能力 达到一名优秀数据分析师的水准,能够真正的通过数据发现业务痛点,赋能业务
以上是我对数据分析师岗位的理解以及个人的短期职业规划

接下来是我入职半年以来的业绩成果总结,主要罗列了7个项目
除了工作时间的业绩,业余时间开展了 linux、excel、hive、前端、git五个课程的学习
主要的进度和投入的时间占比 如图所示 后面将详细介绍

接下来将工作内容按照参与者和负责人的角色划分为两块
作为主要负责人 推进了门店指数、年报、外部调研三个组内项目,临需内容 例如联华公众号小程序 openid,奥莱生日会员member_id,百联生活表调研等就不过多赘述。
关于工作流程 则是是从立项开始,然后获取信息         ,从结论中发现问题,这样一个循环迭代的过程,

关于具体的项目 门店运营指数是以百股下各个门店为主要研究对象,构建门店运营评分指数。共涉及指标23个,聚合成4个方面的得分,最终计算综合分数。项目的产出包括sql库数据表,是一张按月增量更新的分区表,和一张bi看板。具体应用在了第一百货和八佰伴的专项分析中的门店诊断板块
在该项目上的思考,我认为目前的运营指数 更多的是体量上的评估,
在此基础上进一步评估门店运营能力需要控制一些其他的诸如营业面积、所在商圈的价值等变量

关于会员年报项目 主要是为了自动化年报,涵盖了百联通所有会员,以四大业态为主,计算会员相关的各个指标 涉及27类,共813条数据/月,同样是每月更新的分区表,一张bi看板

在这个项目上的思考 主要集中在这张表上,所有的27类不同的指标都在这一张表上,这张表的意义就在于易更新,易迭代 即使是有新的指标过来也只需要在原本的sql语句上增增加一个union

	
第三个项目是大数据行研,了解大数据在行业内的一些应用,整理检索到的信息,形成ppt。
关于零售行业,传统零售行业的核心之一是供应链,但是我们作为全渠道转型的集团 更需要将访客顾客数据联动起来,未来的机会来自与供需链结合 而大数据就是纽带,帮商户做客户所想


第二块工作内容是以参与者为项目角色的专项分析,包括微信粉丝专项,第一八佰伴消费洞察、第一百货专项

三个专项分析中我的工作内容有很多相同的地方,其中工作内容的是第一百货消费洞察,以它为例,该项目是为了了解第一百货门店会员业务的覆盖程度,并且掌握存量会员的特征,包括消费特征和人口特征,并以其为依据,提供可落地的营销方案
主要产出是  1 、2、3、4
我认为这20页ppt是我成长的一个证明,在八佰伴项目中充分发挥了特色技能制作了一个数据包的使用视频。

工作难点主要体现在初期ppt和excel的低效使用;对零售业指标框架的不熟悉
第三是缺乏理论与实践结合的经验
主要成长是相对与难点

在专项分析项目上我的思考或者反思,目前我所提供的更多是业务洞察层面的分析,能够真正落到执行层面的分析还是较少,我觉得主要的原因可能在于对业务的不熟悉,不知道业务或者运营人员想要什么样的指标 另一个是不知道他们可以调控什么样的指标
不知道在研究x对y的影响时,什么样的x是合适的

接下来是自学内容 这部分就简单陈述一下吧,近期愿景是维持学习热情和学习能力
具体自学内容上,学习了git 目前是能够简单使用的阶段,
然后是hive,



关于未来的规划,我这里是按照一年的计划列了五个方向 分成两个板块来看,百分比代表的是未来投入的精力
首先知识储备花费70% 打好基础,
第一是软件编程方向,在完成linux和hive学习的基础上 核心放在spark上,
业务知识上,计划花费大量时间梳理相关业务表逻辑,预计的产出是像m层字典的一张excel表格
                                                                   

\end{document}